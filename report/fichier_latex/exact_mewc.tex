% ----- Exact Algorithm -----

\section{Exact Algorithm}

% ----- Presentation -----

    \subsection{Presentation}

% ----- Fonctionnement -----

\subsection{How it works}

    \hspace*{1cm} \textbf{Etape 1 :}
    \\
    \begin{minipage}{0.5\textwidth}
        \begin{tikzcd} 
            \color{red} \textcircled{1} \arrow[dd, dash, "1"] \arrow[r, dash, "2"] \arrow[ddr, dash, "3"] & \textcircled{2} \arrow[r, dash, "1"] \arrow[ddl, dash, "1"] \arrow[dd, dash, "2"] & \textcircled{5} \arrow[ddl, dash, "1"] \arrow[dd, dash, "2"] \arrow[r, dash, "1"] & \textcircled{7} \arrow[dd, dash, "1"] \arrow[dr, dash, "1"] \\
            & & & & \textcircled{9} \\
            \textcircled{3} \arrow[r, dash, "1"] & \textcircled{4} & \textcircled{6} & \textcircled{8} \arrow[ur, dash, "1"]
        \end{tikzcd}
    \end{minipage}
    \begin{minipage}{0.5\textwidth}
            J'analyse ici le graphique, par exemple ici le graph prendra 1 en entree bla bla bla bla bla bla bla bla bla bla bla bla bla bla bla bla bla bla bla bla bla bla bla bla bla bla bla bla bla bla bla bla bla bla bla bla bla bla bla bla bla bla bla bla bla bla bla bla bla bla bla bla bla bla bla bla bla bla bla bla bla bla bla bla bla bla bla bla bla bla bla bla bla bla bla bla bla bla bla bla bla bla bla bla bla bla bla bla bla bla bla bla bla bla bla bla
    \end{minipage}

% ----- Pseudo - Code -----

\subsection{Pseudo code}

% ----- Complexité -----

\subsection{Complexity}

% ----- Instance -----

\subsection{Instance}

% ----- Experiments -----

\subsection{Experiments}

% ----- Analyse -----

\subsection{Analysis}

\newpage