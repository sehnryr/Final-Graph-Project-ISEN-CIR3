% ----- Introduction -----

\section{Introduction}

% ----- presentation du sujet (je reprends le truc du prof et j'ai add ce que Youn a mis sur discord) -----
\subsection{Presentation of the subject}

    Graph theory is a branch of mathematics that deals with the study of graphs, which are mathematical structures used to model pairwise relationships between objects. Graphs consist of vertices (also called nodes) that are connected by edges. The edges can be directed (one-way) or undirected (two-way) and can have weight. An induced subgraph of a graph is a subgraph that includes all the vertices and edges of the original graph that are present in the subset of vertices chosen for the subgraph. A clique is a complete maximal induced subgraph of a graph. It means, that it will take its most improved form possible and that it cannot be extended to include any more vertices or edges.
    \\ \\
    \begin{minipage}{0.5\textwidth}
        For example, in the following graph, $A$ is a vertex, the link between $A$ and $B$ is an edge. The edge $(CF)$ have a weight of 5. The graph colored in red is an example of induced graph (there are a lot of them). $ABC$ is an induced graph and a clique, the maximal clique here is $ABEC$. 
    \end{minipage}
    \begin{minipage}{0.5\textwidth}
        \begin{center}
            \begin{tikzcd}
                & & A \arrow[dd, dash] \arrow[dl, dash] \arrow[dr, dash] \\
                & \color{red} B \arrow[rr, dash, crossing over] \arrow[dl, dash, red] \arrow[dr, dash, red] & & C \arrow[dl, dash] \arrow[dr, dash, "5"] \\
                \color{red} D \arrow[rr, dash, red] & & \color{red} E \arrow[rr, dash, red] & & \color{red} F 
            \end{tikzcd}
        \end{center}
    \end{minipage}
    \\ \\ \\
    The Maximum Clique Problem (MCP) consists in finding a clique of maximum size. This is a classic graph theory problem that has many-real life applications in various fields such as social networks, chemistry, bioinformatics. It has been studied since the early 1990s \cite{mewcf}.
    \\ \\
    In addition, this problem is $NP$-Hard. It means, this is a problem where the veracity of any solution can be verified quickly (that is, in polynomial time) and a brute-force search algorithm can find a solution by trying all possible solutions. Moreover, NP-complete problems are a type of problem in computer science that are known to be very difficult to solve. These problems are special because if we can find a solution to any one of them, we can use that solution to quickly solve all other problems of the same type. This is because the solutions to NP-complete problems can be easily checked or verified. So, NP-complete problems are kind of like the "benchmark" for how difficult it is to solve a problem. 
    \\ \\
    As a result, it has been studied as a combinatorial optimization problem, being very important in operations research and theoretical computer science.
    \\ \\
    The Maximum Edge Weight Clique Problem (MEWC) consists in finding a clique that maximises the sum of the weights of its edges. Clearly MEWC is also a $NP$-Hard problem since the Maximum Clique Problem is the special case in which all weights are equal.

% ----- Configuration -----

\subsection{Configuration}

    -  Ajouter les configurations que l'on a utilise pour les tests du style la config\\ \\

    On this project, we decide to use \textbf{the C++ language} to developp and implement our algorithms. Nevertheless, a debate took place, especially on the choice of the language. We hesitated between Python and C++. The first one was for us easier to handle and was a tool in which we had more confidence in our ability to use it efficiently. The latter was nevertheless preferred because of its speed of execution, which was an important criterion for the study. We did have some problems with memory allocation, which made us regret this choice at times.  \\ \\
    To share the code between us, we used \textbf{Github}\footnotemark \footnotetext{\url{https://github.com/sehnryr/Final-Graph-Project-ISEN-CIR3}}. It is a web-based platform for version control, collaboration, and sharing of code, as well as a community of developers who contribute to open source projects and share their knowledge. It was a tool that was difficult for some to get used to quickly, especially on the configuration of the project at home, but which brought us a significant gain in efficiency once we had understood how to use it. To share information and communicate between us, we used \textbf{Discord}.
    
    

% ----- Applications -----

\subsection{Example of real-life situations}

    As we said, the MEWC has many real-life applications in various fields such as social networks, chemistry, bioinformatics. We could model this problem with :
    \begin{itemize}
        \item We can model this problem on social \textbf{social networks}, indeed we can represent each \textbf{users} as a \textbf{vertex} in a graph and add an \textbf{edge} between two vertices if the corresponding individuals \textbf{have some kinds of relationships together}. The weight of the edge could represent \textbf{the degree of relationships} between the two users.
        \item Furthermore, we can model the MEWC problem on \textbf{chemistry}, indeed we can represeant each \textbf{} % à finir
    \end{itemize}
    We will take a practical example which could include and involve ISEN students in our future. We will reuse the presented case to illustrate the different algorithms that we will implement later in the report.\\ \\
    \textbf{An example of real-life situations that can be modelled as MEWC} is \ul{the team formation process} during a project, or in the search for a particular social group. 
    \\ \\
    We can imagine that the Student Office of ISEN Nantes is looking to reinforce the video games club of its school. Indeed, the latter has no succession for the following year and is thus led to die if no member presents himself. The future members of the office will have to be in contact with each other during a whole year and it is thus important to find people with common interests so that no tension is formed during their studies. The office has access to the Steam profiles of the students within ISEN (Steam is a video game digital distribution service that gives information about the games played by each one) as well as a record made by the gaming club of the different games played by their members. The fact of playing games in common could bring some people closer, and this makes it a good criterion to create a group that could take over the club because it would share common interests. Otherwise, it would allow to see which games and which group could be present at different events they could organize. Here, forming a team from a group of individuals can be considered as a maximum edge weight clique problem because the goal is to select a subset of individuals such that the members share a common interest that could.
    \\ \\
    To model this problem as a maximum edge weight clique problem, we can represent each \textbf{individual} as a \textbf{vertex} in a graph and add an \textbf{edge} between two vertices if the corresponding individuals \textbf{share atleast one game on Steam}. The weight of the edge could represent \textbf{the number of game} that they have in common.
    \\ \\
    For example, on a small scale we could imagine :
    \\ \\
    \begin{tabular}{|p{10em}|p{35em}|}
        \hline
        \textbf{Students} & \textbf{Game played} \\
        \hline
        Youn & Minecraft, Civilisations, Lost ARK, Among US \\
        \hline
        Martin & Minecraft \\
        \hline
        Valentin & Genshin, Minecraft, Civilisations \\
        \hline
        Maxence & Genshin, Minecraft, Lost ARK, Among US \\
        \hline
        Aubin & CSGO, Genshin, Overwatch, Stardew Valley\\
        \hline
        Dorian & CSGO, Paladins, Overwatch \\
        \hline
        Antoine & League of Legends, Stardew Valley \\
        \hline
        Thomas & League of Legends, The Last of US \\
        \hline
        Alexandre & League of Legends, Dofus \\
        \hline
    \end{tabular}
    \vspace{1\baselineskip} \\
    Which would give us this graph :

    \begin{center}
        \begin{tikzcd} 
            Youn \arrow[dd, dash, "1"] \arrow[r, dash, "2"] \arrow[ddr, dash, "3"] & Valentin \arrow[r, dash, "1"] \arrow[ddl, dash, "1"] \arrow[dd, dash, "2"] & Guillaume \arrow[ddl, dash, "1"] \arrow[dd, dash, "2"] \arrow[r, dash, "1"] & Antoine \arrow[dd, dash, "1"] \arrow[dr, dash, "1"] \\
            & & & & Alexandre \\
            Martin \arrow[r, dash, "1"] & Bastien & Dorian & Thomas \arrow[ur, dash, "1"]
        \end{tikzcd} 
    \end{center}
    
    The goal of the maximum edge weight clique problem in this context would be to find a complete subset of individuals such that the sum of the weights of the edges between the individuals is maximized. In this example, the maximum weight clique would be the clique consisting of nodes Youn, Valentin, Martin, Maxence with a total weight of $10(2+1+1+1+3+2)$.

    % Indeed, we could imagine that Nils Baussé need to make a team for a robotic event in Nantes during the hollidays. As the students of CIR3 do not all live in Nantes during their vacation period, he needs to form a group that is geographically close to each other so that they can meet quickly, do tests and participate in the event to present the project. Here, forming a team from a group of individuals can be considered as a maximum edge weight clique problem because the goal is to select a subset of individuals in order to optimize the global location of the team.
    % \newline

    % To model this problem as a maximum edge weight clique problem, you can represent each individual as a node in a graph and add an edge between two nodes if the corresponding individuals are located within 10 km of each other. The weight of the edge could represent a number proportional to the distance between the two people. The closer they are, the bigger the number would be and the further they are, the smaller the number would be. We will take the weight $w$ as : $$w = \frac{1}{\text{distance between the two individuals}} \times 100$$
    % \newline

    % For example, we could imagine that Youn lives 7km away from Martin, the weigth between these two will be : $$w_{Youn/Martin} = \frac{1}{7} \times 100 = 14.3$$
    % \newline
    % \begin{center}
    %     \begin{tikzcd} 
    %         Youn \arrow[dd, dash, "14.3"] \arrow[r, dash, "23"] \arrow[ddr, dash, "43"] & Valentin \arrow[r, dash, "70"] \arrow[ddl, dash, "15"] \arrow[dd, dash, "100"] & Aubin \arrow[ddl, dash, "85"] \arrow[dd, dash, "15"] \arrow[r, dash, "13"] & Antoine \arrow[dd, dash, "12"] \arrow[dr, dash, "10"] \\
    %         & & & & Alexandre \\
    %         Martin \arrow[r, dash, "16"] & Maxence & Dorian & Thomas \arrow[ur, dash, "12"]
    %     \end{tikzcd} 
    % \end{center}

    % The goal of the maximum edge weight clique problem in this context would be to find a subset of individuals such that the sum of the weights of the edges between the individuals is maximized. In this example, the maximum weight clique would be the clique consisting of nodes Valentin, Aubin, Maxence with a total weight of $255(100+170+85)$.

\newpage