% ----- Analyse -----

\subsection{Analysis}

In figure \ref{fig:constructive_time}, we can observe the polynomial increase in the execution time of the constructive agorithm with the number of vertices for different percentages of connectivity. Note that the theoretical is matching the 75\% connectivity with a constant factor of 4.9.
\bigskip

As we know, the worst case of the algorithm is when the graph is complete. That is, connectivity is at its maximum (100\%). As we can see, increasing the connectivity and the number of vertices of the graph has a chance to increase the complexity of the structure of that graph, which in turn increases the time complexity of the algorithm. 
\newpage

We can notice some factors appearing according to the different number of connectivity. The algorithm takes in average 9 times more time with 75\% than with 25\% and in average 3 times more time with 50\% than with 25\%.
\bigskip

So, the constructive algorithm is good if we want to find an approximative solution for a large number of vertices because on them, we obtain a rather short execution time. Howewer, we will see that it has some boundaries in terms of results obtained. We will detail these points further in the conclusion.