% ----- Presentation -----

\subsection{Presentation}

    A heuristic algorithm is a type of algorithm that uses rules of thumb to try to find an approximate solution to a problem, rather than an exact one. Heuristic algorithms are often used when it is not possible to find an exact solution to a problem, or when an exact solution would be too time-consuming to compute. They are also used when an approximate solution is good enough, or when finding an exact solution is not the primary goal. Heuristic algorithms are commonly used in artificial intelligence, computer science, and other fields. They are often used to solve optimization problems, search problems, and other types of problems where an exact solution is not necessary or practical.
    \\ \\
    A constructive heuristic algorithm is a type of heuristic algorithm that is used to find a solution to a problem by building it incrementally. Constructive heuristics start with a partial solution and gradually add to it until a complete solution is found. They are commonly used to solve optimization problems, where the goal is to find the optimal solution, or the solution that is the best among all possible solutions.
    \\ \\
    \textbf{Constructive heuristics} can be contrasted with other types of heuristics, such as local search heuristics, which try to find a solution by making small changes to an existing solution, or random heuristics, which generate solutions randomly and then choose the best one. Constructive heuristics are often used when it is important to find a solution that is complete and comprehensive, rather than just a local improvement.
    \\ \\
    Examples of some famous problems that are solved using constructive heuristics are the flow shop scheduling, the vehicle routing problem and the open shop problem.