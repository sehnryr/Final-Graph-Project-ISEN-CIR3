\subsection{Analysis}

In figure \ref{fig:exact_time}, we can observe the exponential increase in the
execution time of the exact algorithm with the number of vertices for different
percentages of connectivity. Note that the upper bound appears as a straight
line because of the logarithmic scale but is equal to $n^2\sqrt[3]{3}^n$. The
lower bound is matching the 0\% connectivity with a constant factor of
2.9 $\pm$ 0.09\%. \bigskip

Increasing the connectivity of the graph has a chance to increase the complexity
of the structure of that graph, which in turn increases the time complexity of the
algorithm. The upper bound is the worst case scenario, where the graph maximizes
the number of maximal cliques. The lower bound is the best case scenario, where
the graph is empty. \bigskip

So, the exact algorithm is good if we want to find the exact solution for a few 
vertices. However, if we were to solve this problem for, let's say, 500 vertices,
the exact algorithm would take 8 months to finish and find the solution. And that is
only for 75\% connectivity. If we were to account for the worst case scenario,
the algorithm would take two hundred duovigintillion years ($2\cdot10^{71}$) 
to finish. We can easily conclude that the exact algorithm is not a viable option
for scalability.