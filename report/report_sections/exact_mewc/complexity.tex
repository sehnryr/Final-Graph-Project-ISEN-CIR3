\subsection{Complexity}

Let be a graph $G=(V,E)$ such that $n=|V|$ and $m=|E|$, we can now calculate the
complexity of our algorithm. The cost of attributing a value to a variable should
always be $\mathcal{O}(1)$. \newline

The worst case complexity of the Bron-Kerbosch algorithm is $\mathcal{O}(3^{\frac{n}{3}})$,
we will not prove it here, but it is a well-known fact proven by Moon \& Moser
\cite{on-cliques-in-graphs} that there are at most $3^{\frac{n}{3}}$ maximal cliques
in any $n$-vertex graph. \newline

The complexity of the weight calculation is $\mathcal{O}(n^2)$, because we need
to iterate two times over the vertices of the clique to get every edge and by
extension their cumulative weight. \newline

The worst case complexity of the algorithm is therefore $\mathcal{O}(n^2\times3^{\frac{n}{3}})$,
as we need to iterate through the computed maximal cliques by the Bron-Kerbosch
algorithm and calculate their weight. \newline

Now, the worst case complexity is not the average complexity of the algorithm.
Realistically, the complexity will vary depending on the connectivity of the
graph. Thought calculating its average complexity is not trivial, we can still
give a lower bound of the complexity. \newline

Seeing how the algorithm works by excluding the neighboring vertices of the pivot
vertex from the iteration, the lower bound of the complexity will be the special
case where the graph is empty with no edges connecting the vertices. We can
calculate that complexity by using the following formula:

\begin{equation}
    T(n)=\begin{cases}
        1        & \text{if } n=0 \\
        T(n-1)+n & \text{if } n>0
    \end{cases}
\end{equation}

Base case: $T(0)=1$ \newline
Inductive step: $T(k)=\frac{1}{2}(k^2+k+2)$

\begin{align}
    T(k+1)&=\frac{1}{2}((k+1)^2+(k+1)+2)\\
    &=\frac{1}{2}(k^2+2k+1+k+3)\\
    &=\frac{1}{2}(k^2+k+2)+\frac{1}{2}(2k+2)\\
    &=T(k)+(k+1)
\end{align}

Since we proved that $T(n)=\frac{1}{2}(n^2+n+2)$, we can conclude that the complexity
of the lower bound of the algorithm is $\mathcal{O}(n^2)$. \newline

Also, to match this theoretical complexity, we need to use efficient data structures
with constant insertion, deletion and lookup time. \newline