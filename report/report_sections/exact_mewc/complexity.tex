\subsection{Complexity}

Let be a graph $G=(V,E)$ such that $n=|V|$ and $m=|E|$, we can now calculate the
complexity of our algorithm. The cost of attributing a value to a variable should
always be $\mathcal{O}(1)$. \newline

The worst case complexity of the Bron-Kerbosch algorithm is $\mathcal{O}(3^{\frac{n}{3}})$,
we will not prove it here, but it is a well-known fact proven by Moon \& Moser
\cite{on-cliques-in-graphs} that there are at most $3^{\frac{n}{3}}$ maximal cliques
in any $n$-vertex graph. \newline

The complexity of the weight calculation is $\mathcal{O}(n^2)$, because we need
to iterate two times over the vertices of the clique to get every edge and by
extension their cumulative weight. \newline

The complexity of the algorithm is therefore $\mathcal{O}(n^2\times3^{\frac{n}{3}})$,
as we need to iterate through the computed maximal cliques by the Bron-Kerbosch
algorithm and calculate their weight. \newline

That is the worst case complexity of the algorithm, but realistically, the complexity
will vary depending on the connectivity of the graph. If the graph is not connected,
the Bron-Kerbosch algorithm will complete in $n$ steps and the weight calculation
will be constant as there will be no edge to find. We can easily see the best case
complexity is $\mathcal{O}(n)$. \newline

$m=\frac{n(n-1)}{2}\times c$

The implementation of the algorithm done with efficient data structures permit us
to have a complexity of set insertion, deletion and lookup to be constant in average.
