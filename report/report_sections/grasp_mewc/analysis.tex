% ----- Analyse -----

\subsection{Analysis}

With the figure \ref{fig:grasp_time}, we can comfirm the complexity $\mathcal{O}(n^6)$ we found in the complexity part.
Moreover we can see some similarity with the local search for the execution time. Where they are the same with a constant for the grasp algorithm 
which is the number of iteration we do to find the best solution.
\bigskip

We can observe in the figure \ref{fig:grasp_time}, there is a differences between each percentages of connectivity.
For the 75\% of connectivity, we can observe a factor of $10^3$ when we multiply by 10 the vertices and for the 25\% of connectivity
we have a factor of $10^2$.
\bigskip

Instead of the exact, the Grasp algorithm has a polynomial execution time that is better than the exact and we can use it for graph with a lot of vertices.
In addition, he is more efficient than the local-search algorithm by his iteration that can give better solution than the local-search algorithm.