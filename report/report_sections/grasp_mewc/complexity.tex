% ----- Complexité -----

\subsection{Complexity}

To calculate the complexity of the algorithm, we will consider the complexity of
each part of the algorithm. The first part is the \textbf{greedy randomized heuristic}
and the second one is the \textbf{local search}.
\bigskip

First part: \textbf{Greedy randomized heuristic}
\bigskip

\textsc{getSumAdjacentEdges} is a function that returns the sum of the weights of
the edges adjacent to a vertex. This function iterates over the edges adjacent to
the vertex and sums their weights. The complexity of this function is $\mathcal{O}(m)$.
\bigskip

\textsc{getGamma} is a function that returns the highest sum of adjacent edges
weight in $V_k$. This function iterates over the vertices of $P$ and calls the
\textsc{getSumAdjacentEdges} function. The complexity of this function is
$\mathcal{O}(n\cdot m)$.
\bigskip

\textsc{MakeRCL} is a function that returns the restricted candidate list. This
function calls the \textsc{getGamma} function and iterates over the vertices of
$P$. The complexity of this function is $\mathcal{O}(n\cdot m)$.
\bigskip

\textsc{AdaptGreedyFunction} is a function that adapts the greedy function to the
restricted candidate list. This function iterates over the vertices of $P$ and
checks if they are not adjacent to a vertex. The complexity of this function is
$\mathcal{O}(n)$.
\bigskip

\textsc{ConstructGreedyRandomizedSolution} is a function that constructs a solution
using the greedy randomized heuristic. This iterates over the vertices of $P$ and
calls the \textsc{MakeRCL} function and the \textsc{AdaptGreedyFunction} function.
It also selects a vertex from the RCL and adds it to the clique but as this part 
is constant in complexity, we can ignore it.
The complexity of this function is $\mathcal{O}(n^2\cdot m)$.
\bigskip

Second part: \textbf{Local search}
\bigskip

We know from previously that the complexity of the \textsc{getCliqueWeight} function
is $\mathcal{O}(n^2)$ and the complexity of the \textsc{LocalSearch} function is
$\mathcal{O}(n^5)$. We can then say that the complexity of the \textsc{LocalSearchGrasp}
function is $\mathcal{O}(n^6)$ as it iterates over the vertices of the solution.
\bigskip

Combining the two parts, we can say that the complexity of the \textsc{Grasp} function
is $\mathcal{O}(n^6)$.