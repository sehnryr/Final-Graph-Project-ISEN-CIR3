\subsection{Configuration}

For this project we used \textbf{C++} to develop and implement our algorithms. 
It is a general-purpose programming language that has been used for a long time 
and is still widely used today. It is a compiled language, which means that it 
is translated into machine code before being executed. This allows it to be very 
fast and efficient. It is also a statically typed language, which means that the 
type of each variable must be declared before it is used. This allows the compiler 
to check the type of each variable and to detect errors at compile time. \bigskip

Though C++ is a very powerful language, it is also very complex and has a lot of
features. This can make it difficult to use. That's why we had a debate about
which language to use for this project. Our choices were C++ and Python. We
eventually chose C++ because it is a language that we are all familiar with and
that we have all used before. We also chose it because it is a very powerful
language that allows us to implement very efficient algorithms. \bigskip

The \textbf{Standard Template Library} (STL) is a library of C++ that contains 
a lot of useful classes and functions. It is a very powerful tool that allows 
programmers to work with dynamic data structures without having to implement 
them themselves, and having to deal with memory management. \bigskip

To have the best possible performance, we chose to use \texttt{unordered\_set}
and \texttt{unordered\_map} to store the vertices and edges of the graph. These
data structures are implemented using hash tables, which allows them to have
constant time complexity for insertion, deletion and search. \bigskip

To be able to work on the project efficiently and to be able to share the code
between us, we used \textbf{GitHub}\footnotemark 
\footnotetext{\url{https://github.com/sehnryr/Final-Graph-Project-ISEN-CIR3}}.
It is a web-based platform for version control, collaboration, and sharing of
code, as well as a community of developers who contribute to open source projects
and share their knowledge. It was a tool that was difficult for some to get used
to quickly, especially on the configuration of the project at home, but which
brought us a significant gain in efficiency once we had understood how to use it.
And to share information and communicate between us, we used \textbf{Discord}.
\bigskip

To run our algorithms and to test them, we used a virtual server with \textbf{Debian}
as an operating system. The CPU is an \textbf{AMD EPYC 7282} at 2.8 GHz and with
8 GB of RAM. Although not as powerful as some personal computers, it is still
powerful enough to run our algorithms in a reasonable amount of time and to
perform the tests we needed homogeneously in the background. \bigskip 