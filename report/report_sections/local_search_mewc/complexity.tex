% ----- Complexity -----

\subsection{Complexity}

As for the other algorithms, let be a graph $G=(V,E)$ such that $n=|V|$. The cost 
of attributing a value to a variable should always be $\mathcal{O}(1)$.
\bigskip

The complexity of the clique improvement algorithm is $\mathcal{O}(n^2)$ each time it is called. 
This algorithm is recursive only if it finds an improvement in the clique size. However, each 
time it is called, \textsc{improveClique()} can only find at most $n$ improvements, so it 
recursively calls itself at most $n$ times. So the complexity of \textsc{improveClique()} is 
$\mathcal{O}(n^3)$ when called by a function other than itself.
\bigskip

The complexity of the best neighbor search algorithm is $\mathcal{O}(n^3)$. It goes through 
$n*n$ vertices when computing the vertex that adds a minimum weight, and as shown below the 
complexity of \textsc{improveClique()} is $\mathcal{O}(n^3)$. The complexity of 
\textsc{findNeighbor()} is then $\mathcal{T}(n^2 + n^3) = \mathcal{O}(n^3)$.
\bigskip

The complexity of the initial solution search algorithm is $\mathcal{O}(n^3)$. As for the 
neighbor search algorithm, it is \textsc{improveClique()} that will define the complexity of 
this algorithm because it goes through all the $n$ vertices $2$ times (that is $\mathcal{O}(n)$ 
because get the degree of a vertex is $\mathcal{O}(1)$ as the class is made), and then calls 
the function \textsc{improveClique()} of complexity $\mathcal{O}(n^3)$ to find a maximal clique. 
The complexity of \textsc{findInitialSolution()} is thus $\mathcal{T}(2n + n^3) = \mathcal{O}(n^3)$.
\bigskip

Finally, the local search MEWC algorithm starts by finding an initial solution, which is 
$\mathcal{O}(n^3)$. It gets the weight of this solution ($\mathcal{O}(n)$) and enters a $while$ 
loop and does not exit until the size of $testedVertices$ is equal to the number of vertices of 
the current solution (that is until we have tested to remove each vertex one by one without finding 
a better solution). At each iteration in this $while$ loop, the algorithm calls the function 
\textsc{findNeighbor()} of complexity $\mathcal{O}(n^3)$ and \textsc{getCliqueWeight()} of complexity 
$\mathcal{O}(n)$. It only remains to know how many iterations are done in the $while$ loop. In the 
worst case, the algorithm tries to remove all the vertices each time before finding one that improves 
the solution, and in all the algorithm can only find $n$ better solutions (because each time we find 
a solution, this one has necessarily a size greater than or equal to the previous one, and the more 
we increase the size of the clique, the more the number of possible cliques of this size in the 
graph decreases with a number of $n$ in the case where we have only cliques of $1$ vertex). This 
gives $n*n$ iterations in the $while$ loop. So we have for the local search MEWC algorithm a 
complexity of $\mathcal{T}(n^3 + n + n^2*(n^3 + n)) = \mathcal{O}(n^5)$.
\bigskip

In reality, the actual complexity of this algorithm is only a tiny fraction of $\mathcal{O}(n^5)$. 
Most of the time when calculating the complexity in our case, the $n$ represents the size of the 
actual solution clique instead of the total number of vertices in the graph. Similarly, the number 
of iterations in the $while$ loop is much less than $n^2$ and would be closer to $n$ in most cases, 
except in very special cases.