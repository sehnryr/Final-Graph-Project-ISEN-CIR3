% ----- Presentation -----

\subsection{Presentation}

The local search algorithm is also a heuristic algorithm, like the constructive one. 
So, like the constructive algorithm, the local search will find a solution that may 
or may not be the best one, and we will have no idea how far we are from the real 
solution. The difference between constructive and local search is that constructive 
is based on criteria to find a solution whereas local search will find a first 
solution and visit the "neighbouring" solutions to find an optimal solution. This 
"neighbourhood" criterion is the most important one for this algorithm, it is the 
one that will choose to enter one way or another, and therefore it is the one that 
will give the final solution and define its quality. The neighbour of a solution is 
found by making small changes to the solution found to see if a better solution can 
be obtained. If this is the case, then we start again by looking at the neighbours 
of this new solution up to a certain stopping condition. As can be seen from this 
definition, the local search algorithm will only take the neighbours that have a 
better solution, so if to reach the optimal solution we have to go through a decrease 
in the quality of the solution, then we will never reach this optimal solution. 
Finally, if we consider a graph with the neighbouring solutions on the x-axis and 
the quality of these solutions on the y-axis (in this case the weight of the clique 
found), we will observe that the local search algorithm will only give the local 
maximum, and not the global maximum.


\begin{center}
    \begin{tikzpicture}
        \draw[thick,->] (0,0) -- (10,0) node[anchor=north west] {Solutions};
        \draw[thick,->] (0,0) -- (0,4.5) node[anchor=south east] {Quality (weight)};
        \draw (0,0) .. controls (2,5) and (3,-2) .. (5,3);
        \draw (5,3) .. controls (6,6) and (7,0) .. (10,0);
        \draw[red,thick,->] (2,3) -- (1.5,2.2);
        \node[red] at (2,3.3) {Local maximum};
        \draw[red,thick,->] (7,4.8) -- (5.7,4);
        \node[red] at (7,5.1) {Global maximum};
        \draw[blue,fill] (2.7,1.4) circle (1.5pt);
        \draw[blue,thick,->] (2,0.7) -- (2.5,1.2);
        \node[blue] at (2,0.4) {First solution};
        \draw[green,thick,->] (3.4,1.3) -- (1.8,2.2);
    \end{tikzpicture}
\end{center}

The first thing to do was to find an initial solution that is most likely to be 
close to the best solution. To do this, the algorithm relies on an arbitrary 
criterion that will work in most cases: take the highest degree vertex in the 
graph, its highest degree neighbour and try to form a maximal clique with these 
two vertices. This criterion will allow in most cases to form a rather large 
clique because if we take the highest degree vertex, we have more chance to form 
a larger clique and thus we have more chance to have a clique of great weight.
\bigskip

The second and harder thing to do was to find the neighbourhood criterion: what 
is the neighbour of a maximal clique? After some research, the solution we found 
was to take our current maximum clique, remove a vertex from it (the one that adds 
the least weight to the clique) and see if we can find a new, better maximum 
clique. The problem with this solution for finding neighbours was that we only 
took into account very few cases: we did not take into account the case where we 
had to delete two or more vertices in our clique to find a better one. The 
solution found to solve this problem was to try to remove all sets of $k$ vertices 
and see if we could get a better clique. The new problem with this way of doing 
things was the complexity: to find all sets (or part of sets) of $k$ vertices among 
$n$ (the total number of vertices in the clique), the complexity became $\mathcal{O}(n!)$. 
Now, the principle of a heuristic algorithm like local search is to find an approximate 
solution but with polynomial complexity and $n!$ is anything but polynomial and it 
would take like forever to run this algorithm for a large graph with many of vertices 
and edges.
\bigskip

After several trials running both the factorial complexity algorithm that takes 
sets of vertices to find a neighbouring solution and the one that removes only one 
vertex at a time, we found that both found an approximately similar solution where 
the one that removed only one vertex ran almost instantaneously as opposed to the 
other that took several minutes. As the goal of the local search algorithm is to 
find an approximate solution in a very short time, we decided to keep the first 
neighbourhood criterion found, that is to locate a vertex to try to improve the 
clique found.