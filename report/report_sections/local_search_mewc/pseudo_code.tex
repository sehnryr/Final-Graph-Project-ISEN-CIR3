% ----- Pseudo - Code -----

\subsection{Pseudo code}

\begin{algorithm}[H]
    \caption{Local search MEWC algorithm}
    \label{local-search-mewc-pseudocode}
    \begin{algorithmic}[1]
        \Procedure{LocalSearchMEWC}{$G=(V,E)$}
        
        \State $solution \gets$ \Call{findInitialSolution}{$G$} \Comment{Variable to store the actual solution}
        \State $solutionWeight \gets$ \Call{getCliqueWeight}{$solution$} \Comment{Variable to store the weight of the actual solution}
        \State $testedVertices \gets \emptyset$ \Comment{Variable to store the vertices that have been tested}
        
        \While{$|testedVertices|$ is not $|solution|$}
            \State $neighborSolution \gets$ \Call{findNeighbor}{$G$, $solution$, $testedVertices$} \Comment{Get one of the neighbors of the actual solution (fill $testedVertices$ with the vertex removed in this loop)}
            \State $neighborWeight \gets$ \Call{getCliqueWeight}{$neighborSolution$} \Comment{Get the weight of the neighbor solution}
            \If{$neighborWeight>solutionWeight$}
                \State $solution \gets neighborSolution$
                \State $solutionWeight \gets neighborWeight$
                \State $testedVertices \gets \emptyset$
            \EndIf
        \EndWhile

        \State \Return $solution$
        \EndProcedure
    \end{algorithmic}
\end{algorithm}

\begin{algorithm}[H]
    \caption{Find initial solution algorithm}
    \label{find-initial-solution-pseudocode}
    \begin{algorithmic}[1]
        \Procedure{findInitialSolution}{$G=(V,E)$}
        
        \State $maxDegreeVertex \gets 0$ \Comment{Variable to store the vertex of max degree}
        
        \ForAll{$vertex \in V$}
            \If{d($vertex$) $>$ d($maxDegreeVertex$)}
                \State $maxDegreeVertex \gets vertex$
            \EndIf
        \EndFor

        \State $maxDegreeVertex2 \gets 0$ \Comment{Variable to store the neighbor vertex of $maxDegreeVertex$ of max degree}

        \ForAll{$vertex \in N(maxDegreeVertex)$}
            \If{d($vertex$) $>$ d($maxDegreeVertex2$)}
                \State $maxDegreeVertex2 \gets vertex$
            \EndIf
        \EndFor

        \State $initialSolution \gets \{maxDegreeVertex, maxDegreeVertex2\}$ \Comment{The clique formed by the two vertices of max degree}
        \State \Call{improveClique}{$G$, $initialSolution$} \Comment{Improve $initialSolution$ to have a maximal clique}        
        \State \Return $initialSolution$
        \EndProcedure
    \end{algorithmic}
\end{algorithm}

\begin{algorithm}[H]
    \caption{Find neighbor algorithm}
    \label{find-neighbor-pseudocode}
    \begin{algorithmic}[1]
        \Procedure{findNeighbor}{$G=(V,E)$,$C_i$,$testedVertices$}
        
        \State $minWeightVertex \gets 0$ \Comment{Variable to store the vertex that adds a minimum weight in the clique}
        \State $w_{min} \gets \infty$ \Comment{Variable to store the min weight added by a vertex}

        \ForAll{$v_1 \in V \And v_1 \notin testedVertices$}
            \State $weight \gets 0$ \Comment{Variable to store the weight added by $v_1$}
            \ForAll{$v_2 \in N(maxDegreeVertex) \And v_2 \neq v_1$}
                \State $weight \gets weight + w(v_1,v_2)$
            \EndFor
            \If{$weight < w_{min}$}
                \State $minWeightVertex \gets v_1$
                \State $w_{min} \gets weight$
            \EndIf
        \EndFor

        \State $newClique \gets C_i - \{minWeightVertex\}$ \Comment{Copy the initial clique but remove the tested vertex}
        \State $X \gets \{minWeightVertex\}$ \Comment{Set of vertices that we don't want to be in the new solution}

        \State $improvement \gets$ \Call{improveClique}{$G$, $newClique$, $X$, $w_{min}$} \Comment{$improvement$ will be greater than $0$ if improveClique() find a clique that adds a weight greater than $w_{min}$}

        \If{$improvement <= w_{min}$}
            \State $testedVertices \gets testedVertices + \{minWeightVertex\}$
            \State \Return $C_i$ \Comment{If no better solution have been found, return the initial clique}
        \EndIf

        \State \Return $newClique$
        \EndProcedure
    \end{algorithmic}
\end{algorithm}

\begin{algorithm}[H]
    \caption{Improve clique algorithm}
    \label{improve-clique-pseudocode}
    \begin{algorithmic}[1]
        \Procedure{improveClique}{$G=(V,E)$,$C$,$X$,$w_{min}$,$w_{actual\ improv}$}
        
        \State $C_c \gets C $ \Comment{Variable to store a copy of $clique$}
        \State $w_{improv} \gets 0$ \Comment{Variable to store the weight improvement done by adding a vertex in this iteration}
        \State $total \gets 0$ \Comment{Variable to store the weight improvement made by adding vertices from this iteration}

        \ForAll{$v \in V \textbf{ with } v \notin C_c \And v \notin X$}
            \State $commonNeighbor \gets True$ \Comment{Boolean to know if $v$ is the common neighbor of all vertices of $C$}
            \ForAll{$v_c \in C_c$}
                \If{$(v, v_c) \notin E$}
                    \State $commonNeighbor \gets False$
                    \State \textbf{break}
                \EndIf
            \EndFor
            \If{$commonNeighbor$}
                \ForAll{$v_c \in C_c$}
                    \State $w_{improv} \gets w_{improv} + \text{w(}v,v_c\text{)}$
                \EndFor
                \State $C_c \gets C_c \cup \{v\}$
                \State $total \gets w_{improv}$ + \Call{improveClique}{$G$, $C_c$, $X$, $w_{min}$, $w_{actual\ improv} + w_{improv}$} \Comment{Try improving the clique by adding another vertex}
                \If{$w_{actual\ improv} + w_{improv} \le w_{min}$}
                    \State \Return 0 \Comment{That means that we didn't find a better solution than the old one}
                \Else
                    \State $C \gets C_c$ \Comment{Copy the improved clique int the original one}
                    \State \Return $total$ \Comment{Return the total improvement}
                \EndIf
            \Else
                \State $X \gets X \cup \{v\}$
            \EndIf
        \EndFor
        
        \State \Return $0$ \Comment{Return $0$ if we have no more vertices that are common neighbors of $C$}
        \EndProcedure
    \end{algorithmic}
\end{algorithm}